\section{O que é um Monorepo}

Um Monorepo é uma arquitetura para repositórios de código versionado de projetos, que tem como objetivo centralizar as diversas aplicações e pacotes de uma empresa em um único repositório, podendo elas serem relacionadas ou independentes e gerenciadas por um único ou diversos times dentre desta empresa. 

Algumas empresas utilizam a arquitetura de Monorepo para manter uma única fonte de verdade entre todas as aplicações e pacotes da organização centralizadas. Os Monorepos são normalmente grandes em termos de espaço de armazenamento, alguns dos maiores Monorepos são de grandes empresas como Google, Microsoft, Facebook e Twitter, sendo o Monorepo da Google, em 2016, estimado em cerca de 80TBs (Terabytes), pois houve a decisão logo no início da companhia, que esta seria a sua cultura de desenvolvimento, além de possuírem ferramentas internas que foram desenvolvidas para gerenciar o repositório tendo a cultura da empresa em mente. Este Monorepo é considerado um caso de sucesso e continuam até hoje hospedando cerca de 95\% de suas aplicações neste repositório. 

Existem diversas vantagens em se ter projetos e bibliotecas centralizadas em um único repositório, principalmente quando falamos de compartilhamento de código. Vários projetos web tendem a ser compostos de componentes bases, como botões e campos de texto, em um Monorepo é fácil manter um único componente e disponibilizá-lo para todas as aplicações que vivem dentro do Monorepo, minimizando as chances de código redundante.


Monorepos podem também ser chamados de Repositórios Monolíticos, porém não devem ser confundidos com Arquitetura Monolítica, onde consiste em ter todas as funcionalidades e contexto da aplicação juntas e interligadas, sem uma separação clara entre os contextos.