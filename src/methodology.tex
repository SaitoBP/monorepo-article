\section{Metodologia}

O tipo de pesquisa utilizada para este artigo foi explicativa, usando um projeto de Monorepo para coletar os resultados. Se baseando nos artigos sobre as vantagens e desvantagens de Monorepos \cite{advantages-and-disvantages-of-monorepos} e o trabalho sobre o Monorepo utilizado pela Google \cite{google-monorepo}
Este artigo tem por objetivo demonstrar os conceitos de Monorepo, como podem ser utilizados em projetos modernos, quais suas vantagens e desvantagens, e como podemos garantir que um projeto utilizando a arquitetura de monorepos tenha sucesso.
Durante a construção do projeto de demonstração, alguns problemas foram encontrados, o que levou a troca de ferramenta usada para gerenciar Monorepos. Originalmente foi utilizado a ferramentas Nx \cite{introduction-to-nx}, porem a forma que o Nx trabalha, não condizia com o proposito do projeto, fazendo com que fosse necessario a escolha de outra ferramenta, sendo ela o Turborepo \cite{turborepo}.