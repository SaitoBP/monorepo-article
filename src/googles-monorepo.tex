\section{Monorepo da Google}

Desde o seu início, os desenvolvedores da Google decidiram trabalham centralizando todas as suas aplicações e serviços em um único repositório, e mesmo com o constante crescimento da empresa, e por consequência, o número de desenvolvedores, a arquitetura de Monorepo ainda é utilizada em grande força. Em 2016, foi estimado que o Monorepo da Google se aproximava de 86TBs de dados e cerca de 35 milhões de commits entre os 18 anos de existência do repositório. Devido ao imenso tamanho deste repositório, em 2012 foi necessário o desenvolvimento de ferramentas especializadas para poder gerenciar o repositório. 

A ferramenta desenvolvida para gerenciar o Monorepo recebeu o codinome Piper, de acordo com a Google, não existia uma ferramenta comercialmente disponível que conseguisse lidar com o seu Monorepo. Piper é responsável por hospedar um único repositório implementado sob a infraestrutura da google, e é distribuído entre 10 data centers ao redor do mundo. 