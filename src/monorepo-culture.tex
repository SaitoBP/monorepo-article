\section{Cultura Monorepo}

Como tudo na área de tecnologia, Monorepo não pode ser considerado uma resposta definitiva, uma solução única para todo e qualquer problema, mas sim como uma possível solução para problemas de organização de repositórios e compartilhamento de código. Para que se obtenha sucesso na implementação de Monorepos, é importante que a organização que estiver implementando o Monorepo, esteja preparada para aderir a cultura Monorepo.

O que é a cultura Monorepo? Todos os desenvolvedores estão trabalhando em um único repositório contendo diversos projetos, é importante que haja uma comunicação clara entre os times, para evitar que uma alteração afete o trabalho dos demais. Possuir as aplicações com planos de testes consistentes, para garantir que toda e qualquer mudança não gere problemas para as aplicações. Essas são algumas das bases fundamentais para que a adoção de Monorepos se torne algo produtivo, se os times não estiverem dispostos a trabalharem de acordo com as bases, a implementação do Monorepo que inicialmente tinha o propósito de trazer benefícios para a organização, acaba por se tornar um empecilho 
