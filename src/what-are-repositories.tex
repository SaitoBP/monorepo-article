\section{O que são repositorios}

Um repositório de código é um histórico do código sendo trabalhado, podendo conter outros artefatos além do próprio código, como por exemplo, documentações, exemplos, referencias, entre outros. 

Um repositório de código não é obrigatório para o sucesso de um projeto, porém a utilização de um é considerada essencial entre os desenvolvedores, sendo extremamente raro projetos de larga escala que não possuem um repositório. 

O repositório além de observar as alterações realizadas no código, versiona essas alterações, sendo possível consultá-las futuramente. 

Existem várias formas de se utilizar um repositório, porém a mais comum é através de serviços de hospedagem de repositórios, como GitHub, GitLab e BitBucket, cada um possuindo suas peculiaridades, mas em essência todos tem como Base o Git, que é um sistema de controle de versões distribuído, que é o sistema usado por debaixo dos panos por estes serviços de hospedagem de repositórios.  