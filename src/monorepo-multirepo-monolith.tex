\section{Monorepo Vs. Multirepo Vs. Monolito}

Cada arquitetura de repositórios possui suas vantagens e desvantagens, e não podemos considerar uma solução como sendo a melhor ou pior de todas, cada escopo de projeto ou empresa se adequa a um tipo de arquitetura, e é importante que seja analisado cuidadosamente cada uma antes de se iniciar um projeto, porém sempre é possível migrar de arquitetura caso seja necessário.  

Um Monorepo consiste em um único repositório contendo todas as aplicações de uma organização centralizada. Por ter suas aplicações centralizadas, automaticamente se torna disponível para todos os times, ou seja, qualquer funcionalidade ou componente desenvolvido em uma aplicação, pode ser usado por outro time em outra aplicação, como uma espécie de biblioteca que esta constantemente atualizada, tendo qualquer alteração refletida em todas as demais aplicações que consomem esta funcionalidade, evitando assim as famosas breaking changes, uma vez que quando as mudanças são aplicadas e conflitos são corrigidos antes de ser realizado o commit, não existirão mudanças que quebrem as aplicações. 

Um Multirepo, também conhecido como Polirepo, é uma arquitetura de para repositórios de código versionado de projetos mais comummente utilizado entre os times de desenvolvedores. Uma arquitetura de Multirepo é usada quando existem diversos projetos e diversos times, cada um com seu repositório separando as responsabilidades de cada aplicação em seu repositório. Multirepos não devem ser confundidos com Arquitetura de Micro serviços, uma vez que um projeto dentro de Multirepos podem ou não depender ou estar relacionado a outros projetos. 

O motivo de diversas empresas adotarem a arquitetura de Multirepo é para poder separar as responsabilidades entre times, onde cada produto ou aplicação fica sob responsabilidade de um time, e esse time trabalha apenas em seu repositório, dessa maneira as alterações em um repositório não afetam outro repositório. 

Imagine um cenário onde um time A desenvolve uma aplicação para um mercado, e um time B desenvolve uma aplicação para uma barbearia, ambos são produtos diferentes, porém podem possuir componentes comuns entre eles, como por exemplo, um botão, imagine que tanto o time A quanto o time B criem seus botões, e ambos os botões são idênticos, mesmo design e funcionalidade. Caso seja encontrado um bug, ou caso seja necessário realizar uma modificação nos botões, cada time será responsável por fazer as mesmas alterações em seus respectivos botões. Essa prática gera duplicação de código, fazendo com que tempo e, por consequência, dinheiro seja desperdiçado com um simples botão. Agora imagine que um terceiro time C, se propõem a desenvolver uma biblioteca de botões que serão compartilhados entre os times A e B, será necessário um novo repositório para que o time C possa trabalhar em seus botões, totalizando assim em três repositórios, um para cada time, cada um desses repositórios tem sua própria pipeline de testes, build e publicação. E conforme mais repositórios forem surgindo, mais complexo se torna o gerenciamento e a habilidade de compartilhar código entre os demais projetos. 

Ao implementar a arquitetura de Monorepo, todas as aplicações e bibliotecas ficam no mesmo repositório, fazendo com que tudo que seja desenvolvido dentro do repositório, automaticamente se torna disponível para qualquer time  implementar em seu projeto, além de, toda e qualquer alteração nas bibliotecas, serem refletidas imediatamente em todas as aplicações que as utilizam, evitando as famosas “breaking changes”, uma vez que quando as mudanças são aplicadas antes de ser realizado o commit, não existira uma mudança que quebre as aplicações. 

Já uma arquitetura Monolítica consiste em uma única aplicação que “faz tudo”, sendo mais comum em projetos legado, em uma arquitetura monolítica, conforme for necessário adicionar novas funcionalidades, é criado de forma interligada com a aplicação, dificultando a sua reutilização em outros projetos 