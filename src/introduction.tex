\section{Introdução}

Quando falamos de desenvolvimento, parte importante da construção de um bom software é o armazenamento e versionamento do código, então visando suprir essa necessidade surgem os repositórios, que atendem essa necessidade do projeto ao longo de sua fase de desenvolvimento. Se tratando de repositórios temos três abordagens principais, a abordagem de Monorepo, Multirepo e Monolítica. A abordagem de Multirepo utiliza vários repositórios onde são hospedadas as diversas partes que compõem uma ou mais aplicações. O Monorepo, que surge com uma abordagem oposta, onde todas as partes que compões uma ou mais aplicações, ao invés de ficarem separadas em diferentes repositórios, ficam sob um único repositório, trazendo assim uma série de vantagens, como a possibilidade de reutilização de código, fácil padronização de projetos, e a facilidade de executar todas as partes separadas da aplicação de uma forma única e concisa. 